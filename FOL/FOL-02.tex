\documentclass{beamer}

\usepackage[utf8]{inputenc}
\usepackage{url}

\mode<presentation>
{
  % \usetheme{default}
  % \usetheme{Montpellier}
  % \usetheme{Malmoe}
  % \usetheme{PaloAlto}
  % \usetheme{Berlin}
  % \usetheme{Dresden}
  % \usetheme{Darmstadt}
  % \usetheme{Warsaw}
  % \usetheme{Ilmenau}
  % \usetheme{Frankfurt}
  % \usetheme[secheader]{Madrid}
  % \usetheme{Hannover}
  \usetheme[secheader]{Boadilla}
  % \usetheme{Dresden}
  \setbeamercovered{transparent}
}

\usepackage{textcomp, bbding, wasysym, geometry,graphics,amssymb,amsmath}
\usepackage{enumerate,latexsym,tabular,shapepar}

\usepackage[all,2cell]{xy} \UseAllTwocells \SilentMatrices

\title{LPO - L\'ogica de Primeira Ordem}
\subtitle{Estruturas}
\author{Alexandre Rademaker}

\begin{document}

\frame{\maketitle} 

\frame{\begin{center}
\begin{tabular}{ccc}
 \fbox{Todo} & \fbox{homem} &  \fbox{\'e mortal} \\
$\updownarrow$ & $\updownarrow$ & $\updownarrow$\\
refer\^encia & objeto de um conjunto & propriedade
\end{tabular}\end{center}
\begin{list}{\HandRight}{}
\item Toda refer\^encia ao conjunto dos homens \'e uma refer\^encia ao conjunto dos mortais.
\item Todo elemento pertencente ao conjunto denotado por Homem  pertence ao conjunto denotado por Mortal.
\item $\forall x (H(x)\rightarrow M(x))$
\end{list}
}

\frame{\frametitle{Funç\~oes e Relaç\~oes}
\begin{block}{O pai de Pedro \'e colega de Denise}
 \xymatrix{Pedro \ar[rr]^{Pai}&& Pai(Pedro)}
 \xymatrix{
 &\large\smiley\\
Denise\ar[ur]\ar[r]^{Colega}\ar[dr] &\large\smiley \\
&\large\smiley_{Pai(Pedro)}
}
Colega(Pai(Pedro), Denise)
\end{block}
}

\frame{
\begin{block}{Formalizando}
 \[\xymatrix{&\parbox{5cm}{\centering S\'imbolos L\'ogicos \\($\{\forall, \exists, \wedge, \vee, \neg, \rightarrow\}$ + \\\mbox{vari\'aveis)}}
&\\
 \mbox{Alfabeto}\ar[ur]\ar[dr] & &\\
&\parbox{5cm}{\centering
 s\'imbolos n\~ao l\'ogicos\\
(definidos pelo usu\'ario)
}\ar[dl]\ar[d]\ar[dr]&\\
\mbox{constantes} & \parbox{3cm}{\centering s\'\i mbolos\\ funcionais} & \parbox{4cm}{\centering s\'imbolos\\ predicativos}
}\]
\end{block}
}

\frame{\frametitle{Interpretação e semântica}
Linguagem = <Jo\~ao, Maria, $\mbox{paiDe}^2$, $\mbox{irm\~aoDe}^2$>\par
$I$ associa os elementos da linguagem ao seu significado:\par
$I$(Jo\~ao)$\colon$\textleaf\hspace{2cm}
$I$(Maria)$\colon\heartsuit$\\
$I$(irm\~aoDe) = $\{<\heartsuit,\mbox{\textleaf}>, <\mbox{\textreferencemark}, \mbox{\textleaf}>\ldots\}$\\
$I$(paiDe) = $\{<\sharp,\heartsuit>,<\mbox{\textleaf},\circledast>\ldots\}$\\
\vspace{1.5cm}
Posso atribuir valores verdade à sentenças
}

\frame{
\begin{itemize}[<+->]
 \item $\forall x\forall y A \mbox{ \'e a mesma coisa que } \forall y\forall x A$
 \item $\exists x\exists y A \mbox{ \'e a mesma coisa que } \exists y\exists x A$
 \item $\forall x\exists y A \mbox{ \underline{n\~ao} \'e a mesma coisa que } \exists y \forall x A$
 \item $\exists y\forall x Adora(x, y)$ -- Tem uma pessoa que \'e adorada por todos.
 \item $\forall x\exists y Adora(x, y)$ -- Todo mundo adora algu\'em.
 \item \textit{Dualidade}: cada quantificador pode ser expresso em função do outro:
 \begin{itemize}[<+->]
  \item $\forall x$ Adora($x$, sorvete) \hspace{1.5cm} $\neg\exists x\neg$ Adora($x$, sorvete)
  \item $\exists x$ Adora($x$, jil\'o) \hspace{1.5cm} $\neg\forall x\neg$ Adora($x$, jil\'o)
 \end{itemize}
\end{itemize}
}

\frame{
\begin{exampleblock}{Exercício}
 Reescreva as sentenças abaixo de forma que, se houver a ocorrência de uma negação, ela s\'o ocorrer\'a em um predicado at\^omico:
\begin{itemize}
 \item $\neg(\exists x (P(x)\rightarrow \neg Q(x)))$
 \item $\neg(\forall x (P(x)\rightarrow (Q(x)\vee R(x))))$
 \item $\neg((\exists x P(x))\wedge (\forall x R(x)))$
 \item $\neg(\forall x (Q(x)\rightarrow \exists y (A(y)\wedge R(x, y))))$
 \item $\neg(\forall x\exists y(P(x)\wedge \neg (Q(x)\vee R(x))))$
\end{itemize}
\end{exampleblock}
}

\frame{
\begin{block}{O que quero dizer com a sentença}
 $$\forall x (P(x)\rightarrow Q(x))?$$
\only<2>{essa sentença é verdadeira ou falsa?}
\end{block}
}

\frame{

\begin{list}{\HandRight}{}
\item Sentenças quantificadas s\~ao avaliadas como verdadeiras ou
  falsas em relação a um universo de discurso (ou domínio).

\item Al\'em disso, na medida em que tratamos tamb\'em com objetos e
  suas propriedades ou relações, devemos também indicar claramente a
  qual objeto do universo discurso estamos nos referindo e o
  significado da propriedade ou relação considerada.
  
\item Esse \'e o papel de uma \textit{estrutura}, que podemos pensar
  como sendo uma tradução da linguagem formal para o Portugu\^es.
\end{list}
}

\frame{
Uma estrutura da LPO nos diz:
\begin{itemize}
 \item a que coleção se referem os quantificadores;
 \item o que os outros par\^ametros denotam.
\end{itemize}
\begin{alertblock}{Definição}
Um \textit{par\^ametro} \'e:\\
  um quantificador\\
  um predicado\\
  uma constante\\
  uma função
\end{alertblock}
}

\frame{
\begin{alertblock}{Definição}
 Uma \textit{estrutura} $\mathcal U = <|\mathcal U|, \mathcal P, \mathcal C, \mathcal F>$
para a nossa linguagem da LPO \'e uma função cujo dom\'inio \'e o conjunto de par\^ametros e \'e tal que:
\begin{itemize}
 \item $\mathcal U$ associa a cada quantificador um conjunto n\~ao-vazio $|\mathcal U|$, chamado o universo de $\mathcal U$;
 \item $\mathcal U$ associa a cada predicado $n$-\'ario $P$ uma relação $n$-\'aria $P^{\scriptscriptstyle\mathcal U}\subseteq|\mathcal U|^{\scriptscriptstyle\mathcal U}$,
i.e., $P^{\scriptscriptstyle\mathcal U}$ \'e um conjunto de $n$-uplas de membros do universo;
 \item $\mathcal U$ associa a cada constante $c$ um membro $c^{\scriptscriptstyle\mathcal U}$ do universo $|\mathcal U|$;
 \item $\mathcal U$ associa a cada função $n$-ária $f$ uma operação $n$-ária em $|\mathcal U|$, i.e.,
$f^{\scriptscriptstyle\mathcal U}\colon|\mathcal U|^n\rightarrow|\mathcal U|$
\end{itemize}
\end{alertblock}
}

\begin{frame}

  Na presença de uma estrutura, podemos traduzir sentenças da
  linguagem formal para o Portugu\^es e tentar dizer se essas
  tradu\c{c}\~oes s\~ao verdadeiras ou falsas.

  \begin{exampleblock}{Exemplo}
    Considere a linguagem para a Teoria dos Conjuntos, cujo \'unico
    par\^ametro, al\'em de $\forall$ e $\exists$, \'e $<$:

    \begin{itemize}
    \item $|\mathcal U|\colon =$ o conjunto dos n\'umeros naturais
    \item $\textlbrackdbl <(m,n)\textrbrackdbl^{\scriptscriptstyle\mathcal U}\colon m$ \'e menor que $n$.
      \vspace{1cm}
    \item $\exists x\forall y (\neg<(y,x))$
      \only<2>{
      \item Existe um n\'umero que \'e menor que (ou igual a) todos os outros.
      \item Sabemos que essa sentença é verdadeira.
      \item Dizemos ent\~ao que $\exists x\forall y (\neg<(y,x))$ \'e verdadeira em $\mathcal U$
      \item ou que $\mathcal U$ \'e um \textit{modelo} da sentença.}
    \end{itemize}
  \end{exampleblock}
\end{frame}

\begin{frame}{Exemplo}

  Consider the binary relation \textbf{is-a-factor-of} on the domain
  $\{1,2,3,4,5,6\}$
  
  \begin{itemize}
    \item List all the ordered pairs in the relation.
    \item Display the relation as a directed graph.
    \item Display the relation in tabular form.
    \item Is the relation reflexive? symmetric? transitive?
  \end{itemize}
\end{frame}

\frame{
Dada uma estrutura $\mathcal U$, uma valoração $\mathnormal v$ para essa estrutura, \'e uma função que vai do conjunto das sentenças da linguagem da LPO
em $\{V, F\}$, tal que:
\begin{itemize}
 \item $\mathnormal v(P(t_1, \ldots, t_n)) = \left\{
\begin{array}{c l}   
    V &\mbox{se a propriedade } P^{\scriptscriptstyle\mathcal U} \mbox{ for satisfeita pelos}\\
& \mbox{ pelos objetos } t_1^{\scriptscriptstyle\mathcal U},\ldots, t_n^{\scriptscriptstyle\mathcal U},\mbox{nessa ordem }\\
    F& \mbox{caso contr\'ario }
\end{array}\right.$
  \item $\mathnormal v(\forall A) = \left\{
\begin{array}{c l}   
    V &\mbox{se } \mathnormal v(A[x\leftarrow c])=V \mbox{ para cada objeto } c^{\scriptscriptstyle\mathcal U}\in |\mathcal U|\\
%& \mbox{ do } t_1^{\scriptscriptstyle\mathcal U},\ldots, t_n^{\scriptscriptstyle\mathcal U},\\
    F& \mbox{caso contr\'ario (o que isso significa?) }
\end{array}\right.$
 \item $\mathnormal v(\exists A) = \left\{
\begin{array}{c l}   
    V &\mbox{se } \mathnormal v(A[x\leftarrow c])=V \mbox{ para algum objeto } c^{\scriptscriptstyle\mathcal U}\in |\mathcal U|\\
%& \mbox{ do } t_1^{\scriptscriptstyle\mathcal U},\ldots, t_n^{\scriptscriptstyle\mathcal U},\\
    F& \mbox{caso contr\'ario (o que isso significa?) }
\end{array}\right.$
 \item Os conectivos s\~ao definidos como em LP.
 \end{itemize}
 }

\frame{
\begin{exampleblock}{Exemplo}
$|\mathcal U|$ \'e o conjunto dos n\'umeros naturais\\
$a^{\scriptscriptstyle\mathcal U}\colon 2$\\
$b^{\scriptscriptstyle\mathcal U}\colon 3$\\
$c^{\scriptscriptstyle\mathcal U}\colon 4$\\
% $f^{\scriptscriptstyle\mathcal U}(x)\colon x+ 2$\\
$P^{\scriptscriptstyle\mathcal U}(x)$ \'e par\\
$Q^{\scriptscriptstyle\mathcal U}(x)\colon x$ \'e \'impar\\
$R^{\scriptscriptstyle\mathcal U}(x)\colon x$ \'e primo\\
% $S^{\scriptscriptstyle\mathcal U}(x, y)\colon x$ \'e menor que ou igual a $y$
\end{exampleblock}
Vamos avaliar as seguintes sentenças de acordo com essa interpretação:
\begin{itemize}[<+->]
 \item $\mathnormal v (P(a)) = V$;
 \item $\mathnormal v (P(b)) = F$;
 \item $\mathnormal v (\exists x(P(x) \wedge R(x))) = V$;
 \item $\mathnormal v (\forall x(P (x) \rightarrow R(x))) = F$;
 \item $\mathnormal v (\exists x(P (x) \wedge\neg R(x)) = V$;
 \item $\mathnormal v (\forall x(P (x) \vee Q(x)))  = V$.
\end{itemize}
}

\frame{
\begin{exampleblock}{Exemplo}
Considere\\
 $|\mathcal U|$ \'e o conjunto $\{-1, 0, 1\}$\\
e as interpretações usuais para $`+', `=', `<'$. Determine o valor de verdade para:
\begin{itemize}
 \item $\forall x \exists y (x + y = 0)$;
 \item $ \exists x\forall y (x + y = 0)$;
 \item $\exists x \forall y (x \geq y)$;
 \item $\forall x \forall y \exists z (x + y = z)$

\end{itemize}
\end{exampleblock}}

\frame{\frametitle{Exerc\'icio}
Para cada uma das seguintes f\'ormulas, indique:
\begin{enumerate}
 \item Se \'e uma negação, uma conjunção, uma disjunção, uma implicação, uma f\'ormula univerdal ou uma f\'ormula existencial;
 \item o escopo do quantificador;
 \item as vari\'aveis livres;
 \item se \'e uma sentença.
\end{enumerate}
\begin{itemize}
 \item $\exists x(A(x,y) \wedge B(x)$
 \item $\exists x\exists y(A(x,y) \rightarrow B(x))$
 \item $\neg\exists x\exists yA(x,y) \rightarrow B(x)$
 \item $\forall x\neg \exists y(A(x,y))$
 \item $\exists xA(x,y) \wedge B(x)$
 \item $\exists xA(x,x) \wedge \exists yB(y)$
\end{itemize} 
}

\frame{\frametitle{Exerc\'icio}
Traduza as seguintes sentenças para LPO:
\begin{enumerate}
 \item Todas as coisas s\~ao amargas ou doces.
 \item Ou tudo \'e amargo ou tudo \'e doce.
 \item H\'a algu\'em que \'e amado por todos.
 \item Ningu\'em \'e amado por ningu\'em.
 \item Se algu\'em \'e barulhento, todo mundo fica aborrecido.
\end{enumerate}
}
\end{document}


%%% Local Variables:
%%% mode: latex
%%% TeX-master: t
%%% End:
