\documentclass{beamer}

\usepackage[utf8]{inputenc}
\usepackage{url}

\mode<presentation>
{
  % \usetheme{default}
  % \usetheme{Montpellier}
  % \usetheme{Malmoe}
  % \usetheme{PaloAlto}
  % \usetheme{Berlin}
  % \usetheme{Dresden}
  % \usetheme{Darmstadt}
  % \usetheme{Warsaw}
  % \usetheme{Ilmenau}
  % \usetheme{Frankfurt}
  % \usetheme[secheader]{Madrid}
  % \usetheme{Hannover}
  \usetheme[secheader]{Boadilla}
  % \usetheme{Dresden}
  \setbeamercovered{transparent}
}

\title{LPO - Lógica de Primeira Ordem}
\author{Alexandre Rademaker}

\begin{document}

\frame{\maketitle}

\section{Introdução}

\frame{

  Como simbolizar sentenças como

\begin{itemize}[<+->]
 \item Todo mundo adora l\'ogica.
 \item Tem um restaurante muito bom em Laranjeiras.
 \item Cada um no seu quadrado.
 \item Existe um n\'umero que \'e m\'ultiplo de 3.
\end{itemize}
}

\begin{frame}
\begin{itemize}[<+->]
 \item ``Joana adora l\'ogica'' \'e um caso particular de ``Todo mundo adora l\'ogica.''
 \item ``Jo\~ao adora l\'ogica''\'e um caso particular de ``Todo mundo adora l\'ogica.''
 \item Estou dizendo que existe um universo, um dom\'inio (o conjunto de todas as pessoas no mundo) e que todo mundo nesse
universo adora l\'ogica.
 \item Podemos ver ``adorar l\'ogica'' como uma propriedade. Joana e Jo\~ao, por exemplo, tem essa propriedade.
\end{itemize}
\end{frame}

\frame{
2 \'e menor que 3.\\
\pause
``ser menor que'' é uma relação \\
\pause
assim como ``serem irm\~aos'' na sentença\ ``Joana e Jo\~ao s\~ao irm\~aos.''\\
\pause

Chamaremos de predicado uma expressão utilizada para representar
uma propriedade de um objeto ou relação entre um ou mais objetos.
}

\frame{\title{Simbolizaçao}
 \textit{Existe um número primo.}\\\pause
 \textit{Existe um objeto $x$ (de um certo domínio) tal que $x$ é um número primo.}\\\pause
 Colocando $Primo(x)$ para representar $x$ \'e primo, podemos simbolizar a sentença\ acima da seguinte forma:
 $\exists xP(x)$
}

\frame{\frametitle{Formalizaçao\ -- LPO}
Usaremos os seguintes s\'imbolos:
 \begin{itemize}[<+->]
  \item símbolos lógicos ($\land$, $\lor$, $\to$, $\neg$ etc);
  \item os quantificadores
\begin{description}[<+->]
 \item[universal] $\forall$ (``para todo'')
 \item[existencial] $\exists$ (``existe (pelo menos um)'')
\end{description}
\item predicados ($P, Q, R\ldots$) n-\'arios (ou de grau n) que dependem de:
\item vari\'aveis ($x, y, z\ldots$)
\end{itemize}
\begin{block}{Observaçao}
 Note que um predicado pode ter grau 0.
\end{block}
}

\frame{\frametitle{Exemplos}
$P(x)\colon x$ \'e um n\'umero natural,\\
$Q(x)\colon x$ \'e maior que ou igual a 0,\\
$R(x)\colon x$ \'e primo,\\
$S(x)\colon x$ \'e par,\\
$T(x)\colon x$ \'e negativo.\\
\vspace{1cm}
\only<2,3>{$\forall x(P(x)\rightarrow Q(x))$} \only<3>{-- Todo n\'umero natural \'e maior que ou igual a 0.}
\only<4,5>{$\exists x(P(x) \wedge R(x))$} \only<5>{-- H\'a um n\'umero natural que \'e primo.}
\only<6,7>{$\exists x(P(x) \wedge (\neg S(x)))$} \only<7>{-- Algum n\'umero natural n\~ao \'e par.}
\only<8,9>{$\forall x(P(x) \rightarrow (\neg T(x)) )$} \only<9>{-- Todo n\'umero natural n\~ao \'e negativo.}
}

\frame{
\begin{itemize}[<+->]
 \item $\forall xP(x)$ \'e equivalente a $\forall yP(y)$\newline
que \'e equivalente a $\forall zP(z)$\ldots
 \item $\forall xP(x)\wedge x$ \'e equivalente a $\forall yP(y)\wedge x$\newline
que \'e equivalente a $\forall zP(z)\wedge x$\ldots
 \item Qual \'e a diferença?
 \item Vari\'avel ligada \textit{versus} vari\'avel livre.
\end{itemize}
}

\frame{
\begin{itemize}[<+->]
 \item Chamaremos de \textit{vari\'avel quantificada} uma express\~ao da forma $\forall x$ ou $\exists x$.
 \item A menor f\'ormula que se segue \`a ocorr\^encia de uma vari\'avel quantificada \'e chamada de \textit{escopo} dessa vari\'avel.
 \item $(\forall x {\color{blue}P(x)}) \rightarrow (\forall x {\color{blue}Q(x,y)} \vee R(x))$
 \item $\forall x[{\color{blue} P(x) \rightarrow (\forall x {\color{green}Q(x,y)} \vee R(x))}]$
 \item $(\forall x {\color{blue}P(x)}) \rightarrow (\forall x {(\color{green}Q(x,y) \vee R(x))})$
\end{itemize}
}

\frame{\frametitle{Substituiçao}
\begin{itemize}[<+->]
 \item Indicamos por $A[x\leftarrow a]$ ou $A^x_a$ a substituiçao\ de todas as ocorr\^encias \textbf{livres}
de $x$ por $a$.
 \item $(\forall xP(x))[x\leftarrow a] = \forall xP(x)$
 \item $(\forall xP(x)\wedge x)[x\leftarrow a] = \forall xP(x)\wedge a$
 \item $((\forall x P(x)) \rightarrow (\forall x Q(x,y) \vee R(x)))[x\leftarrow a] =$\\
$(\forall x P(x)) \rightarrow (\forall x Q(x,y) \vee R(a))$
 \item $((\forall x P(x)) \rightarrow (\forall x Q(x,y) \vee R(x)))[y\leftarrow a] = $\\
$(\forall x P(x)) \rightarrow (\forall x Q(x,a) \vee R(x))$
 \item $(\forall x[ P(x) \rightarrow (\forall x Q(x,y) \vee R(x))])[x\leftarrow a] = $\\
$\forall x[ P(x) \rightarrow (\forall x Q(x,y) \vee R(x))]$
\end{itemize}
}

\frame{\frametitle{Exerc\'icio}
\begin{block}{Formalize:}
\begin{itemize}
 \item Todo n\'umero par \'e divisível por 2,
 \item H\'a um n\'umero \'impar que \'e primo,
 \item Algum n\'umero par n\'ao \'e primo,
 \item Todo n\'umero par n\~ao \'e \'impar,
\end{itemize}
\end{block}
\only<2->{De que propriedades precisamos?}
}

\frame{
\begin{block}{}
  $P(x)\colon x$ \'e par\\
  $I(x)\colon x$ \'e \'impar\\
  $D(x)\colon x$ \'e divis\'ivel por 2\\
  $R(x)\colon x$ \'e primo
\end{block}
}

\frame{
\begin{block}{}
 {\color{gray} $P(x)\colon x$ \'e par\\
  $I(x)\colon x$ \'e \'impar\\
  $D(x)\colon x$ \'e divis\'ivel por 2\\
  $R(x)\colon x$ \'e primo}
\vspace{1.5cm}
\begin{itemize}[<+->]
 \item Todo n\'umero par \'e divis\'ivel por 2.
\begin{itemize}[<+->]
 \item $\forall x(P(x)\rightarrow D(x))$
\end{itemize}
 \item Não existe um n\'umero par que n\~ao seja divisível por 2.
\begin{itemize}[<+->]
\item  $\neg\exists x(P(x)\wedge \neg D(x))$
\end{itemize}
\end{itemize}
\end{block}
}

\frame{
\begin{block}{}
 {\color{gray}
  $P(x)\colon x$ \'e par\\
  $I(x)\colon x$ \'e \'impar\\
  $D(x)\colon x$ \'e divis\'ivel por 2\\
  $R(x)\colon x$ \'e primo
}
\vspace{1.5cm}
\begin{itemize}[<+->]
 \item H\'a um n\'umero \'impar que \'e primo.
\begin{itemize}[<+->]
 \item $\exists x(I(x)\wedge R(x))$
\end{itemize}
\item Nem todo n\'umero \'impar n\~ao \'e primo.
\begin{itemize}[<+->]
 \item $\neg\forall x(I(x)\rightarrow \neg R(x))$
\end{itemize}
\end{itemize}
\end{block}
}

\frame{
\begin{block}{} {\color{gray}
  $P(x)\colon x$ \'e par\\
  $I(x)\colon x$ \'e \'impar\\
  $D(x)\colon x$ \'e divis\'ivel por 2\\
  $R(x)\colon x$ \'e primo
}
\vspace{1.5cm}
\begin{itemize}[<+->]
 \item Algum n\'umero par n\'ao \'e primo.
\begin{itemize}[<+->]
 \item $\exists x(P(x)\wedge \neg R(x))$
\end{itemize}
 \item Nem todo n\'umero par \'e primo.
\begin{itemize}[<+->]
 \item $\neg\forall x(P(x)\rightarrow \neg R(x))$
\end{itemize}
\end{itemize}
\end{block}
}

\frame{
\begin{block}{} {\color{gray}
  $P(x)\colon x$ \'e par\\
  $I(x)\colon x$ \'e \'impar\\
  $D(x)\colon x$ \'e divis\'ivel por 2\\
  $R(x)\colon x$ \'e primo
}
\vspace{1.5cm}
\begin{itemize}[<+->]
 \item Todo n\'umero par n\~ao \'e \'impar.
\begin{itemize}[<+->]
 \item $\forall x(P(x)\rightarrow \neg I(x))$
\end{itemize}
 \item Nenhum n\'umero par \'e \'impar.
\begin{itemize}[<+->]
 \item $\neg\exists x(P(x)\wedge I(x))$
\end{itemize}
\end{itemize}
\end{block}
}

\frame{
\begin{block}{}
\begin{itemize}
 \item Todo n\'umero par \'e divis\'ivel por 2,
 \item H\'a um n\'umero \'impar que \'e primo,
 \item Algum n\'umero par n\'ao \'e primo,
 \item Todo n\'umero par n\~ao \'e \'impar,
\end{itemize}
\end{block}
\only<2->{E como fica a negação dessas sentenças?}
}

% \frame{
% \begin{block}{}
% \begin{itemize}
%  \item Todo n\'umero par \'e divis\'ivel por 2,
% \only<2> {\item Existe um n\'umero par que n\~ao \'e divis\'ivel por 2}
% \only<3>  {\item N\~ao \'e o caso que todo  n\'umero par \'e divis\'ivel por 2}
% \end{itemize}
% \end{block}}
% 
% \frame{
% \begin{block}{}%$\forall x(P(x)\rightarrow D(x))$}
%  %\item \only<3>{$\neg\exists x(P(x)\wedge \neg D(x))$
%  \begin{itemize}
%  \item H\'a um n\'umero \'impar que \'e primo,
% \only<2>  {\item Todo n\'umero \'impar n\~ao \'e primo,}
% \only<3>  {\item Nenhum n\'umero \'impar \'e primo}
% \end{itemize}
% \end{block}
% }
% 
% \frame{
% \begin{block}{}
%  \begin{itemize}
%  \item Algum n\'umero par n\'ao \'e primo,
%  \only<2>  { Todo n\'umero par \'e primo,}
%  \only<3>  {\item N\~ao existe um n\'umero par que n\~ao seja primo}
% \end{itemize}
% \end{block}
% }
% 
% \frame{
% \begin{block}{}
%  \begin{itemize}
%  \item Todo n\'umero par n\~ao \'e \'impar,
%   \only<2>{\item Existe um n\'umero par que \'e \'impar,}
% \end{itemize}
% \end{block} \item 
% }

% \frame{Formalizem as negaç{c}\~oes:
% \begin{block}{}
%  \begin{itemize}
%  \item Existe um n\'umero par que n\~ao \'e divis\'ivel por 2
%  \item Todo n\'umero \'impar n\~ao \'e primo, Nenhum n\'umero \'impar \'e primo
%  \item Todo n\'umero par \'e primo, N\~ao existe um n\'umero par que n\~ao seja primo
%  \item Existe um n\'umero par que \'e \'impar,
% \end{itemize}
% \end{block}
% }3

\frame{
\begin{block}{}
 \only<2->{
 {\color{gray}$P(x)\colon x$ \'e par\\
 $I(x)\colon x$ \'e \'impar\\
  $D(x)\colon x$ \'e divis\'ivel por 2\\
  $R(x)\colon x$ \'e primo}
\vspace{1.5cm}
} \begin{itemize}
      \item N\~ao \'e o caso que, todo n\'umero par \'e divis\'ivel por 2
      \only<2->{
	  \begin{itemize}
	    \item $\neg\forall x(P(x)\rightarrow D(x))$
	  \end{itemize}
      }
      \item Existe um n\'umero par que n\~ao \'e divis\'ivel por 2
      \only<3->{
	\begin{itemize}
	  \item $\exists x(P(x)\wedge\neg D(x))$
	\end{itemize}
      }
 \end{itemize}
\end{block}
}

\frame{
\begin{block}{}

\only<2->{
{\color{gray}  $P(x)\colon x$ \'e par\\
  $I(x)\colon x$ \'e \'impar\\
  $D(x)\colon x$ \'e divis\'ivel por 2\\
  $R(x)\colon x$ \'e primo}
}
\vspace{1.5cm}
\begin{itemize}
 \item Nenhum n\'umero \'impar 3 \'e primo.
\only<2->{\begin{itemize}
 \item $\neg\exists x(I(x)\wedge R(x))$
\end{itemize}}
 \item Todo n\'umero \'impar \'e n\~ao primo.
\only<3->{\begin{itemize}
 \item $\forall x(I(x)\rightarrow \neg R(x))$
\end{itemize}}

\end{itemize}
\end{block}
}

\frame{
\begin{block}{}
\only<2->{{\color{gray} 
  $P(x)\colon x$ \'e par\\
  $I(x)\colon x$ \'e \'impar\\
  $D(x)\colon x$ \'e divis\'ivel por 2\\
  $R(x)\colon x$ \'e primo
}}
\vspace{1.5cm}\begin{itemize}
 \item Todo n\'umero par \'e primo.
\only<2->{\begin{itemize}
 \item $\forall x(P(x)\rightarrow R(x))$
\end{itemize}}
 \item N\~ao existe um n\'umero par que n\~ao seja primo.
\only<3>{\begin{itemize}
 \item $\neg\exists x(P(x)\wedge \neg R(x))$
\end{itemize}}
\end{itemize}
\end{block}
}

\frame{
\begin{block}{}
\only<2->{ {\color{gray}  $P(x)\colon x$ \'e par.\\
 $I(x)\colon x$ \'e \'impar.\\
  $D(x)\colon x$ \'e divis\'ivel por 2.\\
  $R(x)\colon x$ \'e primo.
}}
\vspace{1.5cm}\begin{itemize}
 \item  N\~ao \'e o caso que, todo n\'umero par n\~ao \'e \'impar.
\only<2->{\begin{itemize}
 \item $\neg\forall x(P(x)\rightarrow \neg I(x))$
\end{itemize}}
 \item Existe um n\'umero par que \'e \'impar,
\only<3>{\begin{itemize}
 \item $\exists x(P(x)\wedge I(x))$
\end{itemize}}
\end{itemize}
\end{block}
}

\frame{
\begin{block}{Sabemos que $\neg (A\wedge B) \equiv A\rightarrow\neg B$, logo:}
\begin{itemize}
 \item Todo n\'umero par \'e divis\'ivel por 2,
 \item $\forall x(P(x)\rightarrow D(x))$
 \item \only<2>{$\forall x\neg(P(x)\wedge \neg D(x))$}
 \item $\neg\exists x(P(x)\wedge \neg D(x))$
\end{itemize}
\end{block}
}

\frame{
\begin{block}{}
\begin{itemize}
 \item H\'a um n\'umero \'impar que \'e primo,
 \item $\exists x(I(x)\wedge R(x))$
 \item \only<2>{$\exists x\neg(I(x)\rightarrow \neg R(x))$}
 \item $\neg\forall x(I(x)\rightarrow \neg R(x))$
\end{itemize}% \item \only<3>{$\neg\exists x(P(x)\wedge \neg D(x))$
\end{block}
}

\frame{
\begin{block}{}
\begin{itemize}
 \item Algum n\'umero par n\'ao \'e primo,
 \item $\exists x(P(x)\wedge \neg R(x))$
 \item \only<2>{$\exists x\neg(P(x)\rightarrow \neg R(x))$}
 \item $\neg\forall x(P(x)\rightarrow \neg R(x))$
\end{itemize}
\end{block}
}

\frame{
\begin{block}{}
\begin{itemize}
 \item Todo n\'umero par n\~ao \'e \'impar,
 \item $\forall x(P(x)\rightarrow \neg I(x))$
 \item \only<2>{$\forall x\neg(P(x)\wedge I(x))$}
 \item $\neg\exists x(P(x)\wedge I(x))$
\end{itemize}
\end{block}
}

\frame{
\begin{block}{}
 \begin{itemize}[<+->]
\item Qual \'e a relaçao\ entre $\forall$ e $\exists$?
\item $\neg\forall xA$ \'e equivalente a $\exists x\neg A$
\item $\neg\exists xA$ \'e equivalente a $\forall x\neg A$
\item $\neg\forall x\neg A$ \'e equivalente a $\exists x A$
\item $\neg\exists x\neg A$ \'e equivalente a $\forall x A$
\vspace{1.5cm}
\item Ordem:
 \begin{itemize}
 \item $\forall x\exists y(x=5y)$
 \item $\exists y\forall x(x=5y)$
 \end{itemize}
\end{itemize}
\end{block}
}

\frame{
\begin{itemize}
 \item Existe um n\'umero que \'e divis\'ivel por 3.
\only<2->{\begin{itemize}
 \item $D(x,y)\colon x$ \'e divis\'ivel por $y$.
 \item $\exists x(D(x,3))$
\end{itemize}}
 \item A soma de um n\'umero par e 2 \'e um n\'umero par.
\only<3->{\begin{itemize}
  \item $P(x)\colon x$ \'e par.
 \item $S(x, y)\colon$ a soma de $x$ e $y$.
 \item $\forall x( P(x)\rightarrow P(S(x,2)))$
\end{itemize}}
 \item 2 \'e o \'unico n\'umero primo que \'e par.
 \only<4->{\begin{itemize}
  \item $P(x)\colon x$ \'e par.
  \item $R(x)\colon x$  \'e primo.
  \item $=(x,y)\colon x$ \'e igual de $y$.
  \item $\forall x(=(x,2)\leftrightarrow P(x)\wedge R(x))$
\end{itemize}}
\end{itemize}}

\end{document}


%%% Local Variables:
%%% mode: latex
%%% TeX-master: t
%%% End:
